%%%%%%%%%%%%%%%%%%%%%%%%%%%%%%%%%%%%%%%%%%%%%%%%%%%%%%%%%%%%%%%
%
% Welcome to Overleaf --- just edit your LaTeX on the left,
% and we'll compile it for you on the right. If you open the
% 'Share' menu, you can invite other users to edit at the same
% time. See www.overleaf.com/learn for more info. Enjoy!
%
%%%%%%%%%%%%%%%%%%%%%%%%%%%%%%%%%%%%%%%%%%%%%%%%%%%%%%%%%%%%%%%


% Inbuilt themes in beamer
\documentclass{beamer}

% Theme choice:
\usetheme{CambridgeUS}

% Title page details: 
\title{Assignment 1} 
\author{adarsh}
\date{\today}
\logo{\large \LaTeX{}}


\begin{document}

% Title page frame
\begin{frame}
    \titlepage 
\end{frame}

% Remove logo from the next slides
\logo{}


% Outline frame
\begin{frame}{Problem}
   If (k-3),(2k+1),(4k+3) are three consecutive terms of A.P., find the valuee of k

\end{frame}


% Lists frame
\section{solution}
\begin{frame}{Solution}
let a= k-3, b= 2k+1, c =4k+3 are three consecutive terms of A.P.\\

common difference between consecutive terms of A.P. is same\\
\begin{align}
(b-a) & = (c-b)\\
(2k+1)-(k-3)& =(4k+3)-(2k+1)\\
2k+1 -k+3 & = 4k+3 -2k-1\\
k + 4 & = 2k + 2\\
-k  & =  -2\\
k & = 2
\end{align}

\end{frame}


% Blocks frame
\section{Answer}
\begin{frame}{Answer}
k is equal to 2    
\end{frame} 

\end{document}